\documentclass[12pt,a4paper]{article}
\usepackage[utf8]{inputenc}
\usepackage[spanish]{babel}
\usepackage{amsmath}
\usepackage{amsfonts}
\usepackage{amssymb}
\usepackage{graphicx}
\usepackage{listings}
\usepackage{xcolor}
\usepackage{hyperref}
\usepackage{geometry}
\usepackage{fancyhdr}
\usepackage{titlesec}

\geometry{margin=2.5cm}

% Configuración de código Python
\lstset{
    language=Python,
    basicstyle=\ttfamily\small,
    keywordstyle=\color{blue}\bfseries,
    commentstyle=\color{green!60!black},
    stringstyle=\color{red},
    numbers=left,
    numberstyle=\tiny\color{gray},
    stepnumber=1,
    numbersep=5pt,
    backgroundcolor=\color{gray!10},
    frame=single,
    breaklines=true,
    breakatwhitespace=true,
    tabsize=4,
    showstringspaces=false
}

% Encabezado y pie de página
\pagestyle{fancy}
\fancyhf{}
\fancyhead[L]{Sistema Inteligente de Búsqueda de Rutas}
\fancyhead[R]{\thepage}
\fancyfoot[C]{Inteligencia Artificial - Universidad Iberoamericana}

% Títulos
\titleformat{\section}
{\Large\bfseries}
{\thesection}{1em}{}
[\titlerule]

\titleformat{\subsection}
{\large\bfseries}
{\thesubsection}{1em}{}

\title{\textbf{Sistema Inteligente de Búsqueda de Rutas\\en Transporte Masivo\\Basado en Reglas Lógicas y Búsqueda Heurística}}
\author{[Nombres de los integrantes del equipo]}
\date{\today}

\begin{document}

\maketitle

\begin{abstract}
Este documento presenta el desarrollo de un sistema inteligente basado en conocimiento que utiliza reglas lógicas y estrategias de búsqueda heurística para encontrar la mejor ruta entre dos puntos en un sistema de transporte masivo. El sistema implementa algoritmos de búsqueda como A* (A estrella) y BFS (Búsqueda en anchura), junto con una base de conocimiento estructurada mediante reglas lógicas que permiten evaluar y optimizar las rutas considerando factores como tiempo de viaje, costo y número de transbordos.
\end{abstract}

\tableofcontents
\newpage

\section{Introducción}

Los sistemas de transporte masivo en las grandes ciudades requieren de herramientas inteligentes que ayuden a los usuarios a encontrar la mejor ruta entre un punto de origen y un destino. Este proyecto desarrolla un sistema basado en inteligencia artificial que utiliza representación del conocimiento mediante reglas lógicas y algoritmos de búsqueda heurística para resolver este problema.

El sistema implementado permite:
\begin{itemize}
    \item Representar el conocimiento del sistema de transporte mediante reglas lógicas
    \item Encontrar rutas óptimas usando algoritmos de búsqueda heurística
    \item Evaluar rutas considerando múltiples criterios (tiempo, costo, transbordos)
    \item Aplicar reglas de inferencia para determinar conexiones y preferencias
\end{itemize}

\section{Marco Teórico}

\subsection{Representación del Conocimiento mediante Reglas Lógicas}

La representación del conocimiento mediante reglas lógicas permite modelar el dominio del problema de manera declarativa. Una regla lógica tiene la forma:

\begin{equation}
\text{Si } P_1 \land P_2 \land \ldots \land P_n \text{ entonces } C
\end{equation}

Donde $P_1, P_2, \ldots, P_n$ son las premisas y $C$ es la conclusión. En nuestro sistema, las reglas permiten:

\begin{itemize}
    \item Determinar conexiones entre estaciones
    \item Identificar estaciones de transferencia
    \item Evaluar la preferencia de rutas
    \item Ajustar tiempos según condiciones (hora pico)
\end{itemize}

\subsection{Sistemas Basados en Reglas}

Un sistema basado en reglas consiste en:
\begin{enumerate}
    \item \textbf{Base de hechos}: Información sobre el estado actual del sistema
    \item \textbf{Base de reglas}: Conjunto de reglas de producción
    \item \textbf{Motor de inferencia}: Mecanismo que aplica las reglas
\end{enumerate}

En nuestro caso, la base de hechos contiene información sobre estaciones, conexiones y rutas, mientras que las reglas permiten inferir nuevas conexiones y evaluar rutas.

\subsection{Búsqueda Heurística}

Los algoritmos de búsqueda heurística utilizan información adicional (heurísticas) para guiar la exploración del espacio de estados de manera más eficiente.

\subsubsection{Algoritmo A* (A Estrella)}

El algoritmo A* combina las ventajas de la búsqueda en anchura (BFS) y la búsqueda voraz (greedy). Utiliza una función de evaluación:

\begin{equation}
f(n) = g(n) + h(n)
\end{equation}

Donde:
\begin{itemize}
    \item $g(n)$: Costo real desde el nodo inicial hasta el nodo $n$
    \item $h(n)$: Estimación heurística del costo desde el nodo $n$ hasta el objetivo
\end{itemize}

A* garantiza encontrar la solución óptima si la heurística es admisible (nunca sobrestima el costo real).

\subsubsection{Búsqueda en Anchura (BFS)}

BFS explora el grafo nivel por nivel, garantizando encontrar la ruta con menor número de pasos, aunque no necesariamente la más rápida en términos de tiempo o costo.

\section{Metodología}

\subsection{Diseño del Sistema}

El sistema se estructura en los siguientes componentes principales:

\begin{enumerate}
    \item \textbf{Base de Conocimiento}: Almacena estaciones, conexiones y reglas lógicas
    \item \textbf{Motor de Búsqueda}: Implementa algoritmos A* y BFS
    \item \textbf{Sistema de Evaluación}: Aplica reglas para evaluar y comparar rutas
\end{enumerate}

\subsection{Representación del Conocimiento}

El conocimiento se representa mediante:

\begin{itemize}
    \item \textbf{Estaciones}: Nodos del grafo con información de ubicación, tipo y líneas
    \item \textbf{Conexiones}: Aristas del grafo con información de tiempo, costo y línea
    \item \textbf{Reglas}: Lógica para inferir conexiones y evaluar rutas
\end{itemize}

\subsection{Reglas Lógicas Implementadas}

\subsubsection{Regla 1: Conexión en Misma Línea}
\textbf{Premisa}: Dos estaciones pertenecen a la misma línea de transporte.\\
\textbf{Conclusión}: Existe una conexión directa entre ellas.

\subsubsection{Regla 2: Transferencia Permitida}
\textbf{Premisa}: Una estación tiene el servicio de transferencia.\\
\textbf{Conclusión}: Permite cambio de línea sin costo adicional de tiempo significativo.

\subsubsection{Regla 3: Ruta Preferible}
\textbf{Premisa}: Una ruta tiene tiempo menor a 60 minutos y costo menor a 10,000 COP.\\
\textbf{Conclusión}: La ruta es considerada preferible.

\subsubsection{Regla 4: Tiempo Hora Pico}
\textbf{Premisa}: La hora actual está entre 6-9 AM o 5-8 PM.\\
\textbf{Conclusión}: El tiempo de viaje se multiplica por 1.3.

\subsubsection{Regla 5: Sin Transbordos}
\textbf{Premisa}: Una ruta no requiere transbordos.\\
\textbf{Conclusión}: Se aumenta la prioridad de la ruta en 10 puntos.

\section{Implementación}

\subsection{Estructura de Datos}

El sistema utiliza las siguientes estructuras principales:

\begin{lstlisting}
@dataclass
class Estacion:
    nombre: str
    tipo: TipoTransporte
    coordenadas: Tuple[float, float]
    lineas: List[str]
    servicios: List[str]

@dataclass
class Conexion:
    origen: str
    destino: str
    tiempo: int  # minutos
    costo: float  # pesos
    linea: str
    tipo: TipoTransporte

@dataclass
class Ruta:
    estaciones: List[str]
    tiempo_total: int
    costo_total: float
    transbordos: List[str]
    lineas_utilizadas: List[str]
    prioridad: int
\end{lstlisting}

\subsection{Algoritmo A*}

La implementación del algoritmo A* utiliza una cola de prioridad (heap) para mantener los nodos ordenados por su valor $f(n)$:

\begin{lstlisting}
def buscar_ruta_a_estrella(self, origen, destino):
    cola = [(0, 0, origen, [origen], 0, [])]
    visitados = set()
    
    while cola:
        f_score, g_score, actual, ruta, costo, lineas = heapq.heappop(cola)
        
        if actual == destino:
            return construir_ruta(ruta, g_score, costo, lineas)
        
        visitados.add(actual)
        conexiones = self.bc.obtener_conexiones(actual)
        
        for conexion in conexiones:
            nuevo_g = g_score + conexion.tiempo
            h = self.heuristica(conexion.destino, destino)
            nuevo_f = nuevo_g + h
            
            heapq.heappush(cola, (nuevo_f, nuevo_g, ...))
\end{lstlisting}

\subsection{Función Heurística}

La función heurística utiliza la distancia euclidiana entre las coordenadas de las estaciones:

\begin{equation}
h(n) = \sqrt{(x_n - x_{destino})^2 + (y_n - y_{destino})^2} \times 2
\end{equation}

El factor de multiplicación por 2 convierte la distancia en una estimación de tiempo en minutos.

\section{Resultados y Pruebas}

\subsection{Casos de Prueba}

Se implementaron y ejecutaron los siguientes casos de prueba:

\begin{enumerate}
    \item \textbf{Búsqueda básica}: Portal Norte → Centro
    \item \textbf{Búsqueda con transbordos}: Portal Sur → Portal Suba
    \item \textbf{Comparación de algoritmos}: Comparación entre A* y BFS
    \item \textbf{Aplicación de reglas}: Verificación de reglas lógicas
    \item \textbf{Función heurística}: Validación de estimaciones
    \item \textbf{Manejo de errores}: Rutas inexistentes
\end{enumerate}

\subsection{Resultados Obtenidos}

\subsubsection{Ejemplo 1: Ruta Directa}
\textbf{Origen}: Portal Norte\\
\textbf{Destino}: Centro\\
\textbf{Ruta encontrada}: Portal Norte → Calle 100 → Calle 72 → Calle 45 → Centro\\
\textbf{Tiempo}: 20 minutos\\
\textbf{Costo}: 10,000 COP\\
\textbf{Transbordos}: 0

\subsubsection{Ejemplo 2: Ruta con Transbordos}
\textbf{Origen}: Portal Sur\\
\textbf{Destino}: Portal Suba\\
\textbf{Ruta encontrada}: Portal Sur → Kennedy → Centro → Calle 72 → Portal Suba\\
\textbf{Tiempo}: 28 minutos\\
\textbf{Costo}: 10,000 COP\\
\textbf{Transbordos}: 2 (Centro, Calle 72)

\subsection{Análisis de Rendimiento}

El algoritmo A* demostró ser eficiente para encontrar rutas óptimas, explorando significativamente menos nodos que BFS en la mayoría de los casos, gracias a la guía proporcionada por la función heurística.

\section{Conclusiones}

El sistema desarrollado demuestra la efectividad de combinar representación del conocimiento mediante reglas lógicas con algoritmos de búsqueda heurística para resolver problemas de planificación de rutas en sistemas de transporte masivo.

Los principales logros del proyecto incluyen:

\begin{itemize}
    \item Implementación exitosa de una base de conocimiento con reglas lógicas
    \item Desarrollo de algoritmos de búsqueda heurística (A* y BFS)
    \item Sistema funcional que encuentra rutas óptimas considerando múltiples criterios
    \item Validación mediante suite completa de pruebas
\end{itemize}

\subsection{Trabajo Futuro}

Posibles mejoras y extensiones:

\begin{itemize}
    \item Incorporar información en tiempo real sobre el estado del transporte
    \item Implementar aprendizaje automático para ajustar heurísticas
    \item Agregar más criterios de evaluación (accesibilidad, frecuencia)
    \item Desarrollar interfaz gráfica de usuario
    \item Expandir la base de conocimiento con más estaciones y líneas
\end{itemize}

\section{Referencias}

\begin{itemize}
    \item Benítez, R. (2014). \textit{Inteligencia artificial avanzada}. Barcelona: Editorial UOC.
    \begin{itemize}
        \item Capítulo 2: Lógica y representación del conocimiento
        \item Capítulo 3: Sistemas basados en reglas
        \item Capítulo 9: Técnicas basadas en búsquedas heurísticas
    \end{itemize}
    \item Russell, S., \& Norvig, P. (2020). \textit{Artificial Intelligence: A Modern Approach} (4th ed.). Pearson.
    \item Cormen, T. H., Leiserson, C. E., Rivest, R. L., \& Stein, C. (2009). \textit{Introduction to Algorithms} (3rd ed.). MIT Press.
\end{itemize}

\section{Apéndice: Instrucciones de Ejecución}

\subsection{Requisitos del Sistema}
\begin{itemize}
    \item Python 3.7 o superior
    \item No se requieren librerías externas
\end{itemize}

\subsection{Ejecución del Sistema}

Para ejecutar el sistema principal:
\begin{lstlisting}[language=bash]
python sistema_transporte.py
\end{lstlisting}

Para ejecutar las pruebas:
\begin{lstlisting}[language=bash]
python pruebas.py
\end{lstlisting}

\subsection{Repositorio Git}

El código fuente está disponible en:\\
\url{https://github.com/[usuario]/[repositorio]}

\subsection{Video Explicativo}

El video explicativo del proyecto está disponible en:\\
\url{https://[plataforma]/[enlace-al-video]}

\end{document}

